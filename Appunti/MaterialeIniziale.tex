\cleardoublepage

% --- TITLE ---

\begin{titlepage}
	\makeatletter
	\centering
%	{\scshape\LARGE Columbidae University \par}
%	\vspace{1cm}
%	{\scshape\Large Final year project\par}
	\vspace*{1.5cm}
	{\huge\bfseries \@title \par}
	\vspace{2cm}
	{\Large\itshape \@author \par}
	\makeatother
	
	\vfill
\end{titlepage}

% --- COLOPHON ---

\phantomsection
\thispagestyle{empty}

\hfill

\vfill


\noindent Alessandro Candido: \textit{Struttura della materia,}
%tipo di opera,
\textcopyleft\ \DTMMonthname{\the\month} \the\year
\newline

\noindent In copertina: \'Elisabeth Sonrel - \textit{Madre Natura}

(Il reale titolo dell'opera è ignoto all'autore di questo testo, quello presente è frutto di una libera interpretazione. Chiunque si sentisse in dovere di correggerlo è libero di farlo e/o comunicarlo a: \href{mailto:candido.ale@gmail.com}{candido.ale@gmail.com})

\begin{wrapfigure}{R}{0.2\textwidth}
	\centering
	\includegraphics[width=0.2\textwidth]{Licenza/by-sa.pdf}
\end{wrapfigure}
.
\newline

Quest'opera è stata rilasciata con licenza Creative Commons Attribuzione - Condividi allo stesso modo 4.0 Internazionale. Per leggere una copia della licenza visita il sito web \href{http://creativecommons.org/licenses/by-sa/4.0/}{http://creativecommons.org/licenses/by-sa/4.0/}.

% --- DEDICATION ---

\clearpage
\phantomsection
\thispagestyle{empty}
\pdfbookmark{Dedica}{Dedica}

\vspace*{3cm}

\begin{quote}
	Felix qui potuit rerum cognoscere causas, [...] \\
	Fortunatus et ille deos qui novit agrestis, \\ \medskip
	--- P. Vergilius Maro, \textit{Georgicon}
\end{quote}

\medskip

\begin{center}
	Se mi viene in mente qualche idea migliore cambio dedica, ma proprio non mi andava di dedicare delle dispense a qualche mio congiunto, per cui credo di fare la cosa più sensata \textbf{dedicando} queste dispense agli \textbf{studenti di fisica}, gentaccia per lo più... ma in fondo la maggior parte mi stanno simpatici.
\end{center}

% --- INDEX ---

\pdfbookmark{\contentsname}{tableofcontents}
\setcounter{tocdepth}{2}
\tableofcontents
\markboth{\scshape{\contentsname}}{\scshape{\contentsname}}

% --- INTRODUCTION ---

\makeatletter\@openrightfalse\makeatother

\pdfbookmark{Introduzione}{introduzione}

\chapter*{Introduzione}

Queste dispense sono state scritte sulla base del corso di \textit{Struttura della Materia} tenuto dal prof. \textit{A.Tredicucci} nell'anno accademico \textit{2016-2017}.
\newline

A mio parere il corso non ha realmente bisogno di dispense: ognuno può avere la sua opinione sulla chiarezza delle lezioni, ma non c'è dubbio che esse siano ben documentate. Dall'anno in corso il professore ha regolarmente riportato i libri consultati di volta in volta nel \href{http://unimap.unipi.it/registri/dettregistriNEW.php?re=181626::::&ri=12126}{registro delle lezioni}\footnote{Per chi legge a schermo il link è cliccabile, per la stampa non ha senso che io scriva qui l'URL, per cui cercate "Tredicucci registro delle lezioni" su google.} (un utile strumento che si consiglia di consultare), per cui il materiale è disponibile e in abbondanza.
\newline

Ho scritto quindi queste dispense principalmente a mio uso e consumo, per cui insieme a parti fedelmente riportate (in modo possibilmente sintetico, ma spesso privilegiando la completezza) troverete anche pezzi interamente scritti da me: non sto affermando di vantare un contributo diretto agli argomenti trattati, ma piuttosto avvertendo il lettore su ciò che troverà. Le riflessioni presenti sono frutto in parte delle lezioni e in parte delle idee che mi sono venute cercando di chiarire alcuni punti più o meno oscuri (infatti molte riguardano principi e questioni sui fondamenti). Esse sono chiare per me, ma non è detto che lo siano per tutti, per cui siate critici e cercate di farvi la vostra opinione.

\begin{flushright}
	Buon lavoro,\\
	Alessandro
\end{flushright}

\subsubsection*{Attribuzione}
C'è scritto qualcosa nella licenza in seconda pagina (ma si sa che le licenze le leggono in pochi), io però volevo essere un po' più esplicito sulla parte di attribuzione. Non mi interessa granché che il mio nome sia riportato su ogni copia di queste dispense e derivati (cioè: toglietelo pure se preferite), apprezzerei invece che se lo modificaste ci aggiungeste il vostro: non lasciatemi tutta la responsabilità di cose che non ho scritto io!

\makeatletter\@openrighttrue\makeatother